%%%%%%%%%%%%%%%%%%%%%%%%%%%%%%%%%%%%%%%%%%%%%%%%%%%%%%%%%%%%%%%%%%%%%%%%%%%%%%%%%%%%%%%%%%%
%%%%%%%%%%%% Trabalho de final de curso em Eng. de Teleinformatica - UFC %%%%%%%%%%%%%%%%%%
%%%%%%%%%%%%%%%%%%%%%%%%%%%%%%%%%%%%%%%%%%%%%%%%%%%%%%%%%%%%%%%%%%%%%%%%%%%%%%%%%%%%%%%%%%%
%%%%%%%%%%%%%%%%%%%%%%%%%%%%%%% Luiz Camara Neto %%%%%%%%%%%%%%%%%%%%%%%%%%%%%%%%%%%%%%%%%%
%%%%%%%%%%%%%%%%%%%%%%%%%%%%%%%%%%%%%%%%%%%%%%%%%%%%%%%%%%%%%%%%%%%%%%%%%%%%%%%%%%%%%%%%%%%
%%%%%%%%%%%%%%%%%%%%%%%%%%%%%%%%%% Conclusoes %%%%%%%%%%%%%%%%%%%%%%%%%%%%%%%%%%%%%%%%%%%%%
%%%%%%%%%%%%%%%%%%%%%%%%%%%%%%%%%%%%%%%%%%%%%%%%%%%%%%%%%%%%%%%%%%%%%%%%%%%%%%%%%%%%%%%%%%%

\chapter{Conclus\~{o}es}
\thispagestyle{empty}

Este trabalho investigou m\'{e}todos de segmenta\c{c}\~{a}o de vasos da retina e a associa\c{c}\~{a}o entre esta  estrutura segmentada e a sa\'{u}de ocular. Como resultado identificamos os atributos morfol\'{o}gicos que melhor indicavam distor\c{c}\~{a}o na estrutura de vasos e poss\'{i}veis altera\c{c}\~{o}es na sa\'{u}de ocular do paciente. Estes atributos foram avaliados em conjunto com atributos quantitativos para definir uma nova perspectiva de avalia\c{c}\~{a}o de metodologias de segmenta\c{c}\~{a}o de vasos sangu\'{i}neos em imagens de retina onde mais de um conjunto de atributos \'{e} avaliado.

Esta nova abordagem, apresentada neste trabalho, facilita a determina\c{c}\~{a}o da precis\~{a}o das imagens segmentadas, e torna o processo de avalia\c{c}\~{a}o menos dependente da subjetividade humana. Al\'{e}m disso, existe um ganho de desempenho quando comparando com o trabalho manual dos especialistas, pois os testes realizados no software de aplica\c{c}\~{o}es cient\'{i}cas Matlab revelaram que, para uma imagem de tamanho 565 $\times$ 584 pixels, foram necess\'{a}rios apenas 0,3s para que todos os valores dos atributos fossem computados. 

Uma outra contribui\c{c}\~{a}o deste trabalho \'{e} a dispobibiliza\c{c}\~{a}o de uma metodologia de avalia\c{c}\~{a}o de rede segmentada de vasos que poder\'{a} ser utilizada futuramente no projeto de novos algoritmos de segmenta\c{c}\~{a}o. A partir desta metodologia, compara\c{c}\~{o}es mais justas com outros m\'{e}todos existentes podem ser realizadas. 

\section{Perspectivas de Trabalhos Futuros}

Este trabalho mostra a necessidade de seguir investigando novas metodologias de avalia\c{c}\~{a}o da qualidade da segmeta\c{c}\~{a}o de rede de vasos.  Futuramente, ser\~{a}o analisados novos conjuntos de atributos que possuam significativa representatividade na estrutura da rede segmentada. Estes novos atributos podem ser adicionados aos usados na abordagem proposta ou poder\~{a}o ser inclu\'{i}dos em substitui\c{c}\~{a}o a algum atributo usado.

 
